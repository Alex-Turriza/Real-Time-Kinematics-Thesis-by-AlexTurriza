%%%%%%%%%%%%%%%%%%%%%%%%%%%%%%%%%%%%%%%%%
% Masters/Doctoral Thesis 
% LaTeX Template
% Version 2.4 (22/11/16)
%
% This template has been downloaded from:
% http://www.LaTeXTemplates.com
%
% Version 2.x major modifications by:
% Vel (vel@latextemplates.com)
%
% This template is based on a template by:
% Steve Gunn (http://users.ecs.soton.ac.uk/srg/softwaretools/document/templates/)
% Sunil Patel (http://www.sunilpatel.co.uk/thesis-template/)
%
% Template license:
% CC BY-NC-SA 3.0 (http://creativecommons.org/licenses/by-nc-sa/3.0/)
%
%%%%%%%%%%%%%%%%%%%%%%%%%%%%%%%%%%%%%%%%%

%----------------------------------------------------------------------------------------
%	PACKAGES AND OTHER DOCUMENT CONFIGURATIONS
%----------------------------------------------------------------------------------------
\documentclass[
11pt, % The default document font size, options: 10pt, 11pt, 12pt
%oneside, % Two side (alternating margins) for binding by default, uncomment to switch to one side
spanish, % ngerman for German
singlespacing, % Single line spacing, alternatives: onehalfspacing or doublespacing
%draft, % Uncomment to enable draft mode (no pictures, no links, overfull hboxes indicated)
%nolistspacing, % If the document is onehalfspacing or doublespacing, uncomment this to set spacing in lists to single
%liststotoc, % Uncomment to add the list of figures/tables/etc to the table of contents
%toctotoc, % Uncomment to add the main table of contents to the table of contents
%parskip, % Uncomment to add space between paragraphs
%nohyperref, % Uncomment to not load the hyperref package
headsepline, % Uncomment to get a line under the header
%chapterinoneline, % Uncomment to place the chapter title next to the number on one line
%consistentlayout, % Uncomment to change the layout of the declaration, abstract and acknowledgements pages to match the default layout
]{MastersDoctoralThesis} % The class file specifying the document structure
\usepackage[utf8]{inputenc} % Required for inputting international characters
\usepackage{pdfpages}
\usepackage[T1]{fontenc} % Output font encoding for international characters
 
\usepackage{booktabs}
\newcommand{\tabitem}{~~\llap{\textbullet}~~}

\usepackage{palatino} % Use the Palatino font by default

\usepackage{placeins} %FloatBarrier

\usepackage[backend=bibtex,style=numeric,natbib=true]{biblatex} % Use the bibtex backend with the authoryear citation style (which resembles APA)

\addbibresource{Bibliografia.bib} % The filename of the bibliography

\usepackage[autostyle=true]{csquotes} % Required to generate language-dependent quotes in the bibliography

\usepackage{afterpage}

\newcommand\blankpage{%
    \null
    \thispagestyle{empty}%
    \addtocounter{page}{-1}%
    \newpage}
%----------------------------------------------------------------------------------------
%	MARGIN SETTINGS
%----------------------------------------------------------------------------------------

\geometry{
	paper=a4paper, % Change to letterpaper for US letter
	inner=2.5cm, % Inner margin
	outer=3.8cm, % Outer margin
	bindingoffset=.5cm, % Binding offset
	top=1.5cm, % Top margin
	bottom=1.5cm, % Bottom margin
	%showframe, % Uncomment to show how the type block is set on the page
}

%----------------------------------------------------------------------------------------
%	THESIS INFORMATION
%----------------------------------------------------------------------------------------

\thesistitle{Implementación de un Sistema de Navegación Cinética Satelital para Asistir en el Control de Posición de Vehículos Aéreos No Tripulados} % Your thesis title, this is used in the title and abstract, print it elsewhere with \ttitle
\supervisor{Dr. Arturo Espinosa Romero \\Dr. Anabel Martín González} % Your supervisor's name, this is used in the title page, print it elsewhere with \supname
\examiner{} % Your examiner's name, this is not currently used anywhere in the template, print it elsewhere with \examname
\degree{Ingeniero en Computación} % Your degree name, this is used in the title page and abstract, print it elsewhere with \degreename
\author{\textsc{Alex Antonio Turriza Suárez}} % Your name, this is used in the title page and abstract, print it elsewhere with \authorname
\addresses{} % Your address, this is not currently used anywhere in the template, print it elsewhere with \addressname

\subject{Ingeniería} % Your subject area, this is not currently used anywhere in the template, print it elsewhere with \subjectname
\keywords{} % Keywords for your thesis, this is not currently used anywhere in the template, print it elsewhere with \keywordnames
\university{\href{http://www.uady.mx}{Universidad Autónoma de Yucatán}} % Your university's name and URL, this is used in the title page and abstract, print it elsewhere with \univname
\department{\href{http://www.matematicas.uady.mx}{Facultad de Matemáticas}} % Your department's name and URL, this is used in the title page and abstract, print it elsewhere with \deptname
\group{\href{http://researchgroup.university.com}{CLIR}} % Your research group's name and URL, this is used in the title page, print it elsewhere with \groupname
\faculty{\href{http://www.matematicas.uady.mx}{Facultad de Matemáticas}} % Your faculty's name and URL, this is used in the title page and abstract, print it elsewhere with \facname

\AtBeginDocument{
\hypersetup{pdftitle=\ttitle} % Set the PDF's title to your title
\hypersetup{pdfauthor=\authorname} % Set the PDF's author to your name
\hypersetup{pdfkeywords=\keywordnames} % Set the PDF's keywords to your keywords
}

\begin{document}

\frontmatter % Use roman page numbering style (i, ii, iii, iv...) for the pre-content pages

\pagestyle{plain} % Default to the plain heading style until the thesis style is called for the body content

%----------------------------------------------------------------------------------------
%	TITLE PAGE
%----------------------------------------------------------------------------------------

%\begin{titlepage}
%\begin{center}

%\vspace*{.06\textheight}
%{\scshape\LARGE \univname\par}\vspace{1.5cm} % University name
%\textsc{\Large Doctoral Thesis}\\[0.5cm] % Thesis type

%\HRule \\[0.4cm] % Horizontal line
%{\huge \bfseries \ttitle\par}\vspace{0.4cm} % Thesis title
%\HRule \\[1.5cm] % Horizontal line
 
%\begin{minipage}[t]{0.4\textwidth}
%\begin{flushleft} \large
%\emph{Author:}\\
%\href{http://www.johnsmith.com}{\authorname} % Author name - remove the \href bracket to remove the link
%\end{flushleft}
%\end{minipage}
%\begin{minipage}[t]{0.4\textwidth}
%\begin{flushright} \large
%\emph{Supervisor:} \\
%\href{http://www.jamessmith.com}{\supname} % Supervisor name - remove the \href bracket to remove the link  
%\end{flushright}
%\end{minipage}\\[3cm]
 
%\vfill

%\large \textit{A thesis submitted in fulfillment of the requirements\\ for the degree of \degreename}\\[0.3cm] % University requirement text
%\textit{in the}\\[0.4cm]
%\groupname\\\deptname\\[2cm] % Research group name and department name
 
%\vfill

%{\large \today}\\[4cm] % Date
%\includegraphics{Logo} % University/department logo - uncomment to place it
 
%\vfill
%\end{center}
%\end{titlepage}

\includepdf[pages=-]{../PORTADA}
\afterpage{\blankpage}
%----------------------------------------------------------------------------------------
%	DECLARATION PAGE
%----------------------------------------------------------------------------------------

\begin{declaration}
\addchaptertocentry{\authorshipname} % Add the declaration to the table of contents
\noindent Yo, \authorname, declaro que este trabajo de tesis titulado, \enquote{\ttitle} y los resultados presentados son de mi autoría. Declaro que que:

\begin{itemize} 
\item Este trabajo fue realizado durante mi estancia como estudiante de la Facultad.
\item Ninguna parte de esta tesis ha sido previamente presentada para la obtención de un grado u otra promoción en esta Universidad u otra institución.
\item Todas las consultas a trabajos de otras personas han sido claramente atribuidos a sus respectivos autores.
\item Donde se hayan realizado citas textuales del trabajo de otras personas, la fuente siempre estará dada. Con la excepción de dichas citas, esta tesis es totalmente mía.
\end{itemize}
 
\noindent Firma:\\
\rule[0.5em]{25em}{0.5pt} % This prints a line for the signature
 
\noindent Fecha:\\
\rule[0.5em]{25em}{0.5pt} % This prints a line to write the date

\afterpage{\blankpage}

\end{declaration}

%----------------------------------------------------------------------------------------
%	QUOTATION PAGE
%----------------------------------------------------------------------------------------
\newpage
\vspace*{0.4\textheight}

\noindent\enquote{\itshape Si quieres hacer una tarta de manzana desde cero, primero debes inventar un universo.}\bigbreak

\hfill Carl Sagan


\afterpage{\blankpage}
%----------------------------------------------------------------------------------------
%	ABSTRACT PAGE
%----------------------------------------------------------------------------------------

\begin{abstract}
\addchaptertocentry{\abstractname} % Add the abstract to the table of contents
El conocimiento del entorno que nos rodea ha sido tema de interés humano desde antaño, así como la interacción con éste en el día a día, para poder investigar diferentes maneras en cómo se pueden aprovechar los recursos disponibles. Unas herramientas para describir el medio geográfico son los mapas realizados a mano; sin embargo, en lugar de ser una herramienta de exactitud, son sólo una referencia del entorno, dado el gran error derivado de la inexactitud humana en las mediciones.

En el presente trabajo se propone la implementación de un sistema de navegación cinética satelital para vehículos aéreos no tripulados (UAV, por sus siglas en inglés), que sea capaz de proporcionar una precisión muy alta (medible en centímetros) de forma que represente un apoyo en la planeación de rutas y ejecuciones de determinadas tareas cada cierta distancia que permitan la creación de mapas de imágenes aéreas más precisos.

Para alcanzar el objetivo, se trabajará con ayuda de software de procesamiento para RTK, que servirá de apoyo al momento de alcanzar la precisión deseada, además de dispositivos de hardware tales como un par de receptores GPS comunicados de forma inalámbrica, indispensables para implementar una corrección de tipo diferencial.
\end{abstract}

\afterpage{\blankpage}

%----------------------------------------------------------------------------------------
%	ACKNOWLEDGEMENTS
%----------------------------------------------------------------------------------------

\begin{acknowledgements}
\addchaptertocentry{\acknowledgementname} % Add the acknowledgements to the table of contents
\textit{A mis padres} \textbf{Antonio y Lucy}, aquellos que sin importar las circunstancias me apoyaron de principio a fin en esta odisea de estudiar un nivel superior en una ciudad aparte de donde soy originario. A mi \textit{hermana menor} \textbf{Haydeé Krystel}, quien estaba ahí para poder platicar conmigo y animarme en aquellos momentos donde lo necesité en estos años. Mención especial a mis abuelitos y mis tíos, tías, primos y primas, que siempre me apoyaron y me enseñaron muchas cosas y utilidades de sus vidas. \\

\textit{A mis asesores de tesis}, \textbf{Dr. Arturo Espinosa Romero y Dr. Anabel Martín González}, por la innumerable cantidad de conocimientos transmitidos que me serán de utilidad en el resto de la vida, en futuros retos, y por su infinita paciencia ante mis dudas y adversidades especialmente durante la realización de este trabajo. Mención especial a todos mis profesores, no sólo de la universidad. \\

\textit{A mi grupo de cercanos amigos:} Ernesto[\textbf{Nestor9224}], Kevin[\textbf{Kevan357}], Rafa[\textbf{RFer93}], Oscar[\textbf{MadOsc}], George[\textbf{El Tío}], Beto[\textbf{Ax Owl xA}] por estos años reuniéndonos para despejar la mente con los videojuegos como excusa. Junto conmigo, Alex[\textbf{AlexRT07}] somos \textit{los inútiles}, o mejor dicho, [TeamGG]\textbf{Los Gansos Galácticos}. \\

\textit{A mis compañeros de la carrera} por compartir cada momento no sólo en la estancia en la facultad y hacer de esta experiencia universitaria algo más agradable. \\

\textbf{\textit{\Large Muchas gracias a todos.}}

\end{acknowledgements} 
%----------------------------------------------------------------------------------------
%	LIST OF CONTENTS/FIGURES/TABLES PAGES
%----------------------------------------------------------------------------------------

\tableofcontents % Prints the main table of contents

\listoffigures % Prints the list of figures

\listoftables % Prints the list of tables

%----------------------------------------------------------------------------------------
%	ABBREVIATIONS
%----------------------------------------------------------------------------------------

\begin{abbreviations}{ll} % Include a list of abbreviations (a table of two columns)

\textbf{LAH} & \textbf{L}ist \textbf{A}bbreviations \textbf{H}ere\\
\textbf{WSF} & \textbf{W}hat (it) \textbf{S}tands \textbf{F}or\\

\end{abbreviations}

%----------------------------------------------------------------------------------------
%	PHYSICAL CONSTANTS/OTHER DEFINITIONS
%----------------------------------------------------------------------------------------

\begin{constants}{lr@{${}={}$}l} % The list of physical constants is a three column table

% The \SI{}{} command is provided by the siunitx package, see its documentation for instructions on how to use it

Speed of Light & $c_{0}$ & \SI{2.99792458e8}{\meter\per\second} (exact)\\
%Constant Name & $Symbol$ & $Constant Value$ with units\\

\end{constants}

%----------------------------------------------------------------------------------------
%	SYMBOLS
%----------------------------------------------------------------------------------------

\begin{symbols}{lll} % Include a list of Symbols (a three column table)

$a$ & distance & \si{\meter} \\
$P$ & power & \si{\watt} (\si{\joule\per\second}) \\
%Symbol & Name & Unit \\

\addlinespace % Gap to separate the Roman symbols from the Greek

$\omega$ & angular frequency & \si{\radian} \\

\end{symbols}

%----------------------------------------------------------------------------------------
%	DEDICATION
%----------------------------------------------------------------------------------------

\dedicatory{Dedicado a todas aquellas personas que hacen especial mi vida...} 

%----------------------------------------------------------------------------------------
%	THESIS CONTENT - CHAPTERS
%----------------------------------------------------------------------------------------

\mainmatter % Begin numeric (1,2,3...) page numbering

\pagestyle{thesis} % Return the page headers back to the "thesis" style

% Include the chapters of the thesis as separate files from the Chapters folder
% Uncomment the lines as you write the chapters

% Chapter 1

\chapter{Introducción} % Main chapter title

\label{Chap:Intro} % For referencing the chapter elsewhere, use \ref{Intr} 

%----------------------------------------------------------------------------------------

% Define some commands to keep the formatting separted from the content 
\newcommand{\keyword}[1]{\textbf{#1}}
\newcommand{\tabhead}[1]{\textbf{#1}}
\newcommand{\code}[1]{\texttt{#1}}
\newcommand{\file}[1]{\texttt{\bfseries#1}}
\newcommand{\option}[1]{\texttt{\itshape#1}}

%----------------------------------------------------------------------------------------

\section{Preliminares}
En este documento de tesis se presenta la implementación de un sistema de navegación RTK (Real-Time Kinematics, Sistema de Navegación en Tiempo Real) con el apoyo del software RTKLIB en una BeagleBone Black, basándose en \cite{takasu2009development}, para poder realizar control sobre la posición de un aparato que esté en constante movimiento.

\subsection{Descripción del documento}

El presente trabajo se dividirá en las siguientes secciones: 

\begin{itemize}
\item \textbf{Introducción:} Se describen tanto la importancia del tema, los problemas a resolver, estado del arte, y una lista de objetivos a seguir durante el desarrollo del tema.\\
\item \textbf{Marco Teórico:} En este apartado se sientan las bases de funcionamiento de los dispositivos a utilizar, así como información útil acerca de los mismos. Se da una descripción general tanto del hardware como del software utilizado.\\
\item \textbf{Diseño del Hardware y Software:} En esta sección se enlistan diversos recursos implementados en el trabajo, tanto de software como de hardware. Se da una descripción de cada uno y el aporte que tienen a la estructura final durante el funcionamiento.\\
\item \textbf{Diseño del Experimento:} Se describen las condiciones a las que será sometido el prototipo para indicar un adecuado funcionamiento de acuerdo a los objetivos listados en la introducción.\\
\item \textbf{Análisis de Resultados:} Se realiza una evaluación de los resultados obtenidos durante la ejecución de la rutina experimental. Se enfatizan diferencias de los distintos modos de funcionamiento y su impacto en el rendimiento del sistema.\\
\item \textbf{Conclusiones:} En forma de síntesis, se da un resumen de los resultados obtenidos de acuerdo al modo de funcionamiento, las observaciones  y el rendimiento en general del sistema.\\
\end{itemize}

Además, al final se añade una sección de anexos en donde se adjuntan guías de configuración o programación de distintos dispositivos de hardware o de software necesarios para la reproducción del proyecto. 

\section{Importancia del tema}
El conocimiento del medio ha sido tema de interés humano desde antaño. Lo que nos rodea, cómo se interactúa con ello en el día a día, así como poder investigar diferentes maneras en cómo se pueden aprovechar los recursos disponibles. \\

Los mapas realizados a mano, por ejemplo, en lugar de ser una herramienta de exactitud, son solo una referencia del entorno, dado el gran error derivado de la inexactitud humana en las mediciones. \\

Con el avance de la tecnología y el establecimiento de estándares, obtener herramientas de mejor presición ha sido cada vez más viable, y con ello, se puede obtener una mejor descripción del medio, acercándose a lo que en realidad es, y no a cómo lo interpreta la persona que realiza las mediciones. También, el desarrollo de computadores, sensores y actuadores ha permitido un menor sesgo en las mismas. 

\section{Planteamiento del Problema}
Los sistemas de medición actuales poseen cierto grado de incertidumbre, que en ocasiones resulta crítico dado el propósito para el que son utilizados, por ejemplo, en sistemas de tiempo real. \\

\begin{figure}[ht]
\centering
\includegraphics[scale=0.14]{Figures/Pred}
\caption[Posición de celular en mapa.]{Posición aproximada de la ubicación de un celular en un mapa de Google Maps.}
\label{fig:Prec}
\end{figure}

En una planeación de ruta de un sistema de navegación autónomo, una incertidumbre de $n$ metros a la redonda converge en una inexactitud significativa del sistema. El dispositivo puede encontrarse dentro de cualquier punto de la circunferencia de $2n$ metros de diámetro. Revise la figura~\ref{fig:Prec}, donde todo el círculo azul denota la posición aproximada de un dispositivo celular con GPS, y en donde la pequeña elipse roja muestra la posición real en dicho mapa. Además, cuando el sistema no está limitado a sólo seguir una ruta, sino que debe de ejecutar cierta acción cada determinados metros, entonces la incertidumbre se convierte en algo aún más crítico. Es evidente que el sistema no arrojará los resultados esperados debido a la falta de precisión.

\section{Trabajos previos}
En el trabajo \cite{de2011diseno}, se usa un sistema GPS (Global Positioning System, sistema de posicionamiento global, por sus siglas en inglés) para apoyar en la logística de distribución de farmacéuticos, como control de rutas, tiempos y seguridad, mencionando ante todo la capacidad de poder obtener los datos de localización y tiempo, además de la velocidad de desplazamiento de un vehículo. Sin embargo, de acuerdo al trabajo \cite{mendoza2004recomendaciones}, se propone una actualización del ancho de carriles carreteros a una amplitud adecuada máxima de 3.6 m, y dado que un GPS mide su incertidumbre en metros dependiendo del ruido, entonces se está ante un problema que requiere de modelos y cálculos matemáticos extra para reducir dicho error de medición, que podría incluso indicar que el vehículo se encuentra fuera de la cinta asfáltica o circulando en una vía contraria. \\

También, la tesis escrita por \cite{ronnback2000developement}, menciona la implementación de un sistema de navegación inercial INS/GPS para un UAV (Unmanned Aerial Vehicle, vehículo aéreo no tripulado, por sus siglas en inglés). Entre los datos a destacar, la posición de este sistema es estimada con un error de 2 metros con un 95\% de confianza, además de otros datos para asistir al vuelo como velocidad y su altitud. \\


Revisando el desarrollo de \cite{maldonado2010controlador}, los autores afirman que las coordenadas recibidas de un dispositivo GPS solamente son usadas como aproximación a la posición del vehículo, y nunca como un dato sólido que sirviese al computar datos, dada la exactitud mínima de 10 metros. Sin embargo, mencionan limitantes tales como el peso total de la carga del UAV y el consumo de energía causado por la integración de todos los demás sensores, de donde el GPS aporta entonces una cantidad mínima de información y utilidad en general. \\


\subsection{Propuesta}
Dado que en los trabajos previos se presentan algunos inconvenientes con la presición del sistema de navegación inercial de los dispositivos dada la naturaleza del sensor GPS común utilizado, o bien, se requiere una fuerte cantidad de cálculos matemáticos en filtros para aún así seguir recibiendo errores medibles en metros, se propone la realización de un sistema de navegación que utilice una aplicación de cómputo denominada RTKLIB, que nominalmente reduce el error a una escala medible en centímetros a partir de dos señales de GPS, en un sistema portátil de arquitectura ARM, ampliando las posibilidades de aprovechamiento de los datos.

\section{Objetivo}
\subsection{General}
El objetivo general de este trabajo es el diseño y la implementación de un sistema de navegación capaz de conocer su ubicación en un entorno geográfico con una alta precisión para apoyar en proyectos que requieran de dicha información, utilizando un equipo pequeño y portátil.

\subsection{Objetivos específicos}
\begin{itemize}
    \item Configurar la transferencia de datos en UHF (Ultra-High Frequency, ultra alta frecuencia por sus siglas en inglés).
    \item Configuración de los equipos GPS.
    \item Acondicionamiento de una microcomputadora BeagleBone con sus respectivos periféricos de apoyo.
    \item Unificación de todos los módulos.
	\item Evaluación de los datos obtenidos.    
\end{itemize}

\section{Conclusiones}

Un sistema de localización de alta precisión incluso para objetivos móviles tiene un alto potencial dadas las posibles aplicaciones. Como un sensor en un sistema de control, GPS posee un error muy grande para emplearse en trabajos que requieran exactitud y estabilidad, como la toma de fotografías aéreas. Mediante corrección diferencial de tipo Real-Time Kinematics, se puede reducir dicha incertidumbre a niveles de escasos decímetros. El objetivo de este trabajo es el de implementar dicha corrección en un sistema móvil controlado por una BeagleBone Black.\\

En el capítulo~\ref{Chap:Marco}, se expone la teoría que se encuentra detrás de la metodología implementada en el proyecto de tesis, partiendo desde la definición de un sistema GPS, pasando por su modo de funcionamiento y las principales causas de error en sus mediciones. También, se presenta una breve descripción del funcionamiento de una comunicación inalámbrica, del software y del hardware utilizado.
% Chapter 2

\chapter{Marco Teórico} % Main chapter title

\label{Chap:Marco} % For referencing the chapter elsewhere, use \ref{Chapter2} 

%----------------------------------------------------------------------------------------

% Define some commands to keep the formatting separated from the content 
%\newcommand{\keyword}[1]{\textbf{#1}}
%\newcommand{\tabhead}[1]{\textbf{#1}}
%\newcommand{\code}[1]{\texttt{#1}}
%\newcommand{\file}[1]{\texttt{\bfseries#1}}
%\newcommand{\option}[1]{\texttt{\itshape#1}}

%----------------------------------------------------------------------------------------

\section{El sistema de Posicionamiento Global - GPS}

GPS fue iniciado en 1973 para su uso con fines militares por los Estados Unidos de Norteamérica. Su objetivo principal es la determinación de las coordenadas espaciales bajo una referencia mundial. Para dichos propósitos, se necesita una recepción de señales de un mínimo de cuatro satélites, cuyas coordenadas son plenamente conocidas
\textbf{[Huerta, E.; Mangiaterra, A.; Noguera, G. 2005]}.

\subsection{La constelación NAVSTAR}

La constelación NAVSTAR está conformada por los satélites que se utilizan para triangular la posición. Se propuso que fueran 24 satélites situados en seis planos de cuatro satélites cada uno con cobertura en toda la Tierra. \\

En 1983, el presidente Ronald Reagan permitió el uso civil del sistema, no sin antes añadir un error (llamado \textit{Disponibilidad Selectiva}) para evitar que fuese tan preciso. \\

En el año 2000, el presidente Bill Clinton retiró el error intencionado \textbf{[Zambrano Termal, Jhon. 2014]}.

\subsection{Dispositivos GPS}
Los dispositivos GPS son aparatos que obtienen la información de los satélites y realizan un procesamiento del mismo para ubicarse en el marco de referencia terrestre. \\

Existen GPS de distintos tamaños y marcas.

\newpage

\begin{table}[htb]
\begin{center}
\caption{Características del GPS Navspark RAW.}
\begin{tabular}{|l|}
	\hline
	        \ \ \ \ \ \ \ \ \ \ \ \ \ \ \ \ \ \ \ \ \ \ \ \ \ \ \ \ \ \ \ \ \ \ \ \ \ \ \ \ \ \ \ \ \ \ \ \ \ \ \textbf{Navspark RAW GPS} \\
	\hline
	%\begin{figure}[H]%Example image
		\\  \ \ \ \ \ \ \ \ \ \ \ \ \ \ \ \ \ \ \ \ \ \ \ \ \ \ \ \ \ \ \ \ \ \ \ \ \ \ \ \ \includegraphics[width=0.37\linewidth]{Figures/NavGPS}
	%	\caption{GPS Navspark.}
		\label{fig:nsraw}\\
	%\end{figure} \\
	
	\textbf{Características:}\\
	%\begin{itemize}
		\tabitem Procesador: 100MHz 32bit LEON3 Sparc-V8 + IEEE-754 Compliant FPU.\\
		\tabitem 17 Digital I/O.\\
		\tabitem GPS con actualización de hasta 20 Hz.\\
		\tabitem Rango de operación: (h $<$ 18000 msnm) \& (v $<$ 515 m/s).\\
		\tabitem Precisión de hasta 2.5 metros.\\
		\tabitem Consumo: 15 ma @ 3.3V.\\
	%\end{itemize} \\
	\hline
\end{tabular}
\end{center}
\end{table}

\begin{table}[!htb]
\begin{center}
\caption{Características del GPS Ublox C94 M8P.}
\begin{tabular}{|l|}
	\hline
	\ \ \ \ \ \ \ \ \ \ \ \ \ \ \ \ \ \ \ \ \ \ \ \ \ \ \ \ \ \ \ \ \ \ \ \ \ \ \ \ \ \ \ \ \ \ \ \ \ \textbf{Ublox C94 M8P GPS}\\
	\hline
	%\begin{figure}[H] % Example image
	      \ \ \ \ \ \ \ \ \ \ \ \ \ \ \ \ \ \ \ \ \ \ \ \ \ \ \ \ \ \ \ \ \ \ \ \ \ \ \ \ \includegraphics[width=0.37\linewidth]{Figures/ublox}\footnotemark
	%\caption{GPS Ublox. \footnotemark}
	\label{fig:ubx} \\
	%\end{figure} \\
	\textbf{Características: }\\

	%\begin{itemize}
		\tabitem 72 channel u-blox M8 engine GPS L1C/A, GLONASS L1OF, BeiDou B1.\\
		\tabitem GPS con actualizaciones de hasta 10 Hz.\\
		\tabitem Rango de operación: (h $<$ 50000 msnm) \& (v $<$ 500 m/s).\\
		\tabitem Precisión de hasta 2.5 metros.\\
		\tabitem Consumo: 23 ma @ 3.3V.\\
		\tabitem \textcolor{blue}{Incluye módulo de radiofrecuencia.}\\
	%\end{itemize} \\
	\hline
\end{tabular}
\end{center}
\end{table}

\footnotetext{Imagen alojada en www.u-blox.com}

\FloatBarrier

\newpage

\subsection{Fundamento matemático}

\subsection{Causas de error}

Como en todo sistema de medición, es probable que un cálculo de una posición se vea afectado por cualquiera de las siguientes:

\begin{itemize}
	\item Satélites.
	\item Atmósfera.
	\item Rutas múltiples.
	\item Receptor.
\end{itemize} 

\subsubsection{Error causado por el satélite}

Los satélites incorporan relojes atómicos de gran exactitud. Como el tiempo es crítico al momento de la triangulación de cualquier dispositivo, un error de apenas un nanosegundo equivale a un error en distancia de 30 cm. Los relojes atómicos acumulan un error de esta magnitud cada tres años. \\

También, al error en la posición de los satélites sobre sus órbitas se le atribuye un error de 2.1 metros.

\subsubsection{Error causado por la atmósfera}

Las señales de radio que comunican a los satélites y los receptores deben atravesar a la atmósfera a través de considerables kilómetros. El sólo hecho de atravesar partículas cargadas en la ionósfera\footnotemark y entrar en contacto con el vapor de agua de la tropósfera causa variaciones de velocidad en la transmisión. \\

El error en distancia atribuible a esta etapa es de 4 metros.

\footnotetext{El error obtenido en la ionósfera puede eliminarse utilizando receptores de frecuencia doble L1 y L2.}

\subsubsection{Error causado por rutas múltiples}

 El sistema está diseñado para funcionar idealmente a cielo abierto. Sin embargo, en condiciones normales, esto no puede ser del todo reproducible, ya sea por el uso en áreas forestales o áreas urbanas. \\
 
Normalmente, la señal directa llega primero al receptor, y después, arriban las que proceden de las rutas múltiples. Esta diferencia de tiempo ocasiona otro error de medición. \\

Este mismo efecto era visible en las señales análogas de televisión, en las imágenes dobles.

Las nuevas antenas exteriores pueden filtrar este efecto.

\subsubsection{Error causado por el receptor}

Así como los satélites, los receptores cuentan con sus propios relojes. Debido a costos y dimensiones, éstos no pueden ser atómicos y por tanto son menos exactos, siendo otra fuente de error en la medición.

El error asociado a esta causa suele ser de 0.5 metros \textbf{[Fallas, 2002]}.

\section{Comunicación inalámbrica}

Se dice que dos dispositivos se comunican de forma \textbf{inalámbrica} cuando éstos interactúan sin un contacto sólido entre sus masas. \\

En los aparatos electrónicos, se suelen usar ondas de radiofrecuencia, que a su vez se subclasifican dependiendo de la frecuencia a la que son emitidas.

A menor frecuencia, se obtiene una gran cobertura pero la capacidad se ve mermada. Conforme se aumenta la frecuencia, se pierde capacidad de cobertura pero la carga que puede tener una banda, aumenta. Se dice entonces que es una relación inversamente proporcional.

Actualmente no existe algún organismo internacional que regule las frecuencias utilizadas...

\begin{table}[!htb]
\begin{center}
\caption{Bandas de frecuencia en comunicación inalámbrica.}
\begin{tabular}{|l|l|l|l|l|}
	\hline
	\textbf{País/Región} & \textbf{LF} & \textbf{HF} & \textbf{UHF} & \textbf{Microondas}\\
	\hline
	USA & 125-134 KHz & 13.56 MHz & 902-928 MHz & 2400-2483.5 MHz\\& & & & 5725-5850 MHz \\
	\hline
	Europa & 125-134 KHz & 13.56 MHz & 865-868 MHz & 2.45 GHz \\
	\hline
	Japón & 125-134 KHz & 13.56 MHz & No permitida & 2.45 GHz \\
	\hline
	China & 125-134 KHz & 13.56 MHz & No permitida & 2446-2454 MHz \\
	\hline
\end{tabular}
\end{center}
\end{table}

Continuar choro de Tapia, D. I., Cueli, J. R., García, Ó., Corchado, J. M., Bajo, J., \& Saavedra, A. (2007). Identificacion por radiofrecuencia: fundamentos y aplicaciones. Proceedings de las primeras Jornadas Científicas sobre RFID. Ciudad Real, Spain, 1-5.

\subsection{Protocolo ZigBee}

El estándar IEEE 802.15.4, mejor conocido como ZigBee, es una especificación para aplicaciones de control remoto para cualquier equipo que requiera de un bajo costo y un bajo consumo de potencia en entornos reducidos. ZigBee puede funcionar a tres bandas de frecuencia diferentes: 868 MHz, 915 MHz y 2.4 GHz. \\

Los módulos XBEE, fabricados por Digi International siguen el protocolo ZigBee. De entre todos esos módulos, destacan los de la serie PRO, ya que poseen una mayor potencia en la señal y en consecuencia, pueden hasta duplicar la capacidad de alcance en la distancia de transmisión. \\

El módulo requiere una alimentación que va desde los 2.8 V hasta los 3.4 V.

\subsection{Módulo UHF del Ublox C94-M8P GPS}

Los dispositivos GPS Ublox C94-M8P están diseñados para trabajar en pares y vienen incorporados con módulos de comunicación inalámbrica que operan en los 915 MHz de frecuencia en el continente americano. \\

En él, se puede configurar el envío y recepción de diferentes contenidos. El manual de Ublox recomienda usar el formato RTCM3, que será explicado más adelante.

\section{Sistemas de cómputo embebidos}

Los Sistemas Embebidos son sistemas programables, que realizan tareas específicas determinadas por el usuario, con el objetivo de optimizar los procesos para mejorar su desempeño y eficiencia, reduciendo tamaño y costos de producción. \\

Se caracterizan por el bajo consumo de energía. Está compuesto por tres componentes principales: Procesador, Dispositivos de almacenamiento y Periféricos [Caballero Paz, Andrés. 2014].

\subsection{BeagleBone}

La BeagleBone es una plataforma de desarrollo de bajo costo. \\

Desarrollada por la BeagleBone Foundation de los Estados Unidos, una fundación sin ánimos de lucro, cuyo objetivo es la promoción de hardware y software de código abierto para el desarrollo de sistemas embebidos. \\

Por su diseño, posee una arquitectura ARM, que posee soporte de varias distribuciones Linux [Coronado Vallés, Jorge. 2014]. \\

Por su condición de hardware y software abierto, tanto sus esquemáticos del hardware, como las códigos fuente de su software están disponibles a todos los usuarios. \\

De las ventajas que posee una arquitectura como la del BeagleBone, basada en microprocesadores, es que su uso conduce a plataformas más poderosas, capaces de realizar tareas de gran carga computacional [Coley, Gerald. 2013].

\section{Real Time Kinematics}

Se le llama Sistema de Navegación Cinética Satelital en Tiempo Real (RTK, Real Time Kinematics por sus siglas en inglés), a las correcciones de la señal del GPS basadas en las señales L1 y L2. \\

Todo está en torno al siguiente supuesto: \\

Se tienen dos receptores a una distancia de pocos km entre sí. En esta condición, se podría esperar que los errores causados por el reloj atómico del satélite, por la ionósfera y la tropósfera afectarían de igual manera y con la misma magnitud a ambos receptores, por su proximidad. Si la posición exacta de uno de los receptores es conocida, entonces esta información puede ser usada para determinar el error asociado a las lecturas de dicho receptor y después aplicar la corrección al otro dispositivo. \\

Al GPS cuya posición es conocida recibe el nombre de \textbf{receptor base} y el segundo es llamado \textbf{receptor móvil}. La estación base calcula la distancia entre cada uno de los satélites de los que recibe señal y su posición (también conocida) para determinar el error asociado a la medición de distancia. Esa información la envía al receptor móvil, quien aplica la corrección hacia dicho satélite, obteniendo así un conocimiento más exacto sobre su posición [Fallas, 2002]. \\

Con todas las correcciones aplicadas de forma ideal, se alcanza una precisión mejor a los 10 cm [Cerrato Miranda, Javier. 2011].

\section{Formato RTCM3}

Este formato es un estándar internacional para transmitir datos de posicionamiento en tiempo real. \\

El mensaje contendrá los siguientes datos: \\

\begin{table}[!htb]
\begin{center}
\caption{Contenido del mensaje en formato RTCM3.1}
\begin{tabular}{|l|}
	\hline
	\textbf{Mensaje RTCM3.1}\\
	\hline
	\tabitem \textbf{1005:} (X,Y,Z) Coordenadas fijas de la antena. \\
	\tabitem \textbf{1077:} Observaciones de GPS. \\
	\tabitem \textbf{1087:} Observaciones de GLONASS.\footnotemark \\
	\hline
\end{tabular}
\end{center}
\end{table}

\footnotetext{Homólogo ruso del sistema americano GPS.}

\section{RTKLIB}

Desarrollado por Tomoji Takasu, RTKLIB es un paquete de programas de código abierto para posicionamiento tanto estándar como preciso. Soporta varios modos de posicionamiento tales como: Simple, Diferencial, Cinemático, entre otros. \\

En todos sus modos soporta tanto procesamiento en tiempo real así como postprocesamiento. 
% Chapter 2

\chapter{Diseño del hardware}
\label{Chap:DisHard} % For referencing the chapter elsewhere, use \ref{Chapter2} 

%----------------------------------------------------------------------------------------

\section{Introducción}

Lorem ipsum dolor...

\section{Prototipo}

La interconexión general del sistema quedará como sigue: Es posible observar a la estación base o de tierra en la parte inferior izquierda de la imagen. La estación móvil o Rover se encontrará navegando libremente en un espacio cercano en función a la potencia de los módulos de radiofrecuencia usados para comunicar a ambas plataformas.

\section{Descripción del funcionamiento}

A la estación base se le asignará un conjunto de coordenadas fijas de latitud, longitud y altura sobre el nivel del mar, que es donde ella deberá ser colocada. El GPS realizará las triangulaciones necesarias para obtener su posición estimada, también en coordenadas, a través de los satélites. Así, comparando a las coordenadas previamente fijadas de su posición exacta con las obtenidas de las triangulaciones, la estación base calculará el
error que existe en ese instante en las mediciones y procederá a informar, mediante los módulos de comunicación de radiofrecuencia, a la estación móvil, para que proceda a hacer las correcciones necesarias y obtener una aproximación depurada de su posición y sus movimientos. \\

Se realizó un experimento que consistió en trazar y seguir una determinada ruta para poder observar las diferencias entre un sistema RTK y uno con el GPS triangulando por sí solo. \\

También fueron colocadas marcas en el suelo para poder tener una guía precisa del camino a seguir y repetir en ambos experimentos. \\

Se usaron las coordenadas que Google Maps proporciona, como referencia. De este modo, los resultados irán conforme a los mapas de este servicio. \\

Al final, la ruta a seguir quedaría de este modo:

\section{Descripción del experimento}

Lorem ipsum dolor...

\section{Conclusión}

Lorem ipsum dolor...

En el capítulo siguiente...
% Chapter 2

\chapter{Resultados} % Main chapter title

\label{Chap:Res} % For referencing the chapter elsewhere, use \ref{Chapter2} 

\section{Introducción}

En este capítulo se presentarán los procedimientos requeridos para llevar a cabo el experimento descrito en el capítulo anterior, y se expondrán los resultados obtenidos de realizar las mediciones. Posteriormente, se mostrarán gráficas de los datos obtenidos con una explicación de sus componentes y su respectiva interpretación. \\

\section{Dibujo del plano}

Para el trazo de la figura cuadrado, que será de 15 metros por lado, se utilizaron dos varas unidas por una fracción de rafia como las de la figura~\ref{fig:HerTraz}, utilizados para el trazado de las circunferencias necesarias.\\

\begin{figure}[H]
\centering
\includegraphics[width=0.95\textwidth]{Figures/Herr}
\caption[Herramientas de trazado.]{Herramientas de trazado.}
\label{fig:HerTraz}
\end{figure}

Para garantizar que se trace una línea recta en cada uno de los lados de la figura, se utilizó una cuerda de nailon de 17 metros de longitud, que una vez tensa, permitió realizar las marcas equidistantes cada 3 metros con un flexómetro, signos en el suelo como los de la figura~\ref{fig:MarEq}.

\begin{figure}[H]
\centering
\includegraphics[width=0.95\textwidth]{Figures/Equid}
\caption[Marcas equidistantes en el suelo.]{Marcas equidistantes en el suelo.}
\label{fig:MarEq}
\end{figure}

\section{Mediciones}
Una vez finalizada la separación en 5 segmentos por cada lado de la figura, de 3 metros de separación entre ellos, se obtienen 6 puntos de medición; tras ser configuradas ambas estaciones, se procedió a realizar las mediciones de las coordenadas en los puntos. Se decidió tomar muestras tanto en el modo Real-Time Kinematics como con el modo de un solo GPS sin retroalimentación de una estación base para comparar resultados. En ambos modos, se recorrió el trazado de la figura dos veces con la antena, en dos sentidos diferentes, llegando a obtener dos mediciones de coordenadas por punto en cada modo de funcionamiento.\\

\begin{figure}[H]
\centering
\includegraphics[width=0.95\textwidth]{Figures/Dispers}
\caption[Diagrama de dispersión de datos registrados en las marcas del lado sur de la figura.]{Diagrama de dispersión de datos registrados en las marcas del lado sur de la figura.}
\label{fig:Disp}
\end{figure}

En la figura~\ref{fig:Disp} se comparan las mediciones realizadas en las marcas del lado sur de la figura, mediante una regresión lineal utilizando los puntos del segundo recorrido en el modo Real-Time Kinematics. Los cuadrados azules marcan el segundo recorrido de muestreo de la figura realizado en el modo Real-Time Kinematics. Los triángulos amarillos marcan las muestras del primer recorrido de muestreo en el mismo modo. El rombo rojo y el triángulo verde marcan el mismo orden pero del modo con un sólo GPS sin retroalimentación de posicionamiento.\\

Las muestras realizadas en el modo Real-Time Kinematics muestran, en su segundo recorrido, marcado con cuadros azules, un mayor ajuste a la línea recta de la figura, donde el valor del coeficiente de correlación múltiple $R^{2} \approx 0.997$\footnotemark sugiere un alto ajuste a una línea recta. En el primer recorrido, marcado con triángulos amarillos, se puede observar la misma tendencia pero con un menor ajuste. El resto de puntos, marcados con círculos rojos y triángulos verdes, son en un modo sin Real-Time Kinematics y se puede observar que el ajuste a una línea recta es mucho menor que el modo RTK, llegando a tener variaciones bastante considerables respecto a dicho método.\\

\footnotetext{El valor de $R^{2}$ varía de 0 a 1. Mientras más cerca esté su valor de la unidad, se sugiere una correlación más alta a la recta en este caso.}

En el modo Real-Time Kinematics, en cada muestra, existe un valor denominado \textit{ratio} que indica la calidad de la señal de los satélites y de las mediciones del GPS en el \textit{rover} de forma proporcional. En otras palabras, mientras más alto sea el valor de ratio, mucha mejor será la corrección de la señal de GPS. En la figura~\ref{fig:Ratio} se observa cómo en las primeras muestras, el valor del \textit{ratio} oscilaba en valores bajos (se consideran valores bajos aquellos que están debajo de 3). Conforme el tiempo avanzaba, se captaban más satélites y, como consecuencia, el valor del \textit{ratio} ascendió a valores altos. Este comportamiento se ajusta a la gráfica de dispersión en la figura~\ref{fig:Disp}, que muestra que en el primer recorrido, denotado con un triángulo amarillo, el primer punto muestreado (el que está más a la izquierda) estuvo algo más separado de la recta que el resto de muestras en el mismo modo de funcionamiento.

\begin{figure}[H]
\centering
\includegraphics[width=0.95\textwidth]{Figures/Ratio}
\caption[Gráfica del ratio (confianza de los datos presentados).]{Gráfica del ratio (confianza de los datos presentados).}
\label{fig:Ratio}
\end{figure}

Finalmente, las coordenadas obtenidas durante la rutina se muestran en el mapa en las figuras~\ref{fig:NoRtkRes}~y~\ref{fig:RtkRes}, siendo la primera la solución sin retroalimentación y la segunda la solución con RTK.

\begin{figure}[H]
\centering
\includegraphics[width=0.95\textwidth]{Figures/NoRtkRes}
\caption[Muestras obtenidas sin Real-Time Kinematics.]{Muestras obtenidas por el sistema sin Real-Time Kinematics.}
\label{fig:NoRtkRes}
\end{figure}

\begin{figure}[H]
\centering
\includegraphics[width=0.95\textwidth]{Figures/RtkRes}
\caption[Muestras obtenidas por el sistema con Real-Time Kinematics.]{Muestras obtenidas por el sistema \textbf{con Real-Time Kinematics}.}
\label{fig:RtkRes}
\end{figure}

Nótese cómo en la solución sin retroalimentación (figura~\ref{fig:NoRtkRes}) se muestran variaciones muy altas en las líneas rectas que conforman el cuadrado, y las muestras se ven afectadas notoriamente en el lado este de la figura. Por otro lado, en la solución con RTK se muestra un mucho mejor ajuste a los puntos marcados y se observa un mejor acoplamiento a la forma del cuadrado.

\section{Conclusión}
Tras el análisis de los resultados, se determinó que la solución con Real-Time Kinematics ayuda a mantener la estabilidad de las mediciones de GPS, pudiendo determinar mejor la posición de los puntos al realizar un recorrido de una rutina previamente estructurada. 
% Chapter 2

\chapter{Conclusiones} % Main chapter title

\label{Conc} % For referencing the chapter elsewhere, use \ref{Chapter2} 

Pepe 

%----------------------------------------------------------------------------------------
%	THESIS CONTENT - APPENDICES
%----------------------------------------------------------------------------------------

\appendix % Cue to tell LaTeX that the following "chapters" are Appendices

% Include the appendices of the thesis as separate files from the Appendices folder
% Uncomment the lines as you write the Appendices

\includepdf[pages=-]{Apendices/nfs}
\includepdf[pages=-]{Apendices/qemu}
\includepdf[pages=-]{Apendices/xbeeGuia}

%%%%%%%%%%%%%%%%%%%%%%%%%%%%%%%%%%%%%%%%%%
% Simple Sectioned Essay Template
% LaTeX Template
%
% This template has been downloaded from:
% http://www.latextemplates.com
%
% Note:
% The \lipsum[#] commands throughout this template generate dummy text
% to fill the template out. These commands should all be removed when 
% writing essay content.
%
%%%%%%%%%%%%%%%%%%%%%%%%%%%%%%%%%%%%%%%%%

%----------------------------------------------------------------------------------------
%	PACKAGES AND OTHER DOCUMENT CONFIGURATIONS
%----------------------------------------------------------------------------------------

%\documentclass[12pt]{article} % Default font size is 12pt, it can be changed here

%\usepackage{geometry} % Required to change the page size to A4
%\geometry{letterpaper} % Set the page size to be A4 as opposed to the default US Letter

%\usepackage{graphicx} % Required for including pictures

%\usepackage{listings} %Para comandos bash Linux

%\usepackage{float} % Allows putting an [H] in \begin{figure} to specify the exact location of the figure
%\usepackage{wrapfig} % Allows in-line images such as the example fish picture

%\usepackage{color} %textos de colores

%\usepackage[spanish]{babel}

%\usepackage[utf8]{inputenc} %Uso de acentos directamente

%\usepackage{hyperref}

%\linespread{1.2} % Line spacing

%\setlength\parindent{0pt} % Uncomment to remove all indentation from paragraphsre

%\graphicspath{{Pictures/}} % Specifies the directory where pictures are stored

%\begin{document}

%----------------------------------------------------------------------------------------
%	TITLE PAGE
%----------------------------------------------------------------------------------------

%\begin{titlepage}

%\newcommand{\HRule}{\rule{\linewidth}{0.5mm}} % Defines a new command for the horizontal lines, change thickness here

%\center % Center everything on the page

%\textsc{\LARGE Universidad Autónoma de Yucatán}\\[1.5cm] % Name of your university/college
%\textsc{\Large Facultad de Matemáticas}\\[0.5cm] % Major heading such as course name
%\textsc{\large Anexo de tesis de Alex Antonio Turriza Suárez}\\[0.5cm] % Minor heading such as course title

%\HRule \\[0.4cm]
%{ \huge \bfseries Configuración de un Sistema de Archivos en Red [NFS] entre una PC x86-64 y una BeagleBone Black}\\[0.4cm] % Title of your document
%\HRule \\[1.5cm]
%\begin{minipage}{0.5\textwidth}
%\begin{flushleft} \large
%\emph{Autor:}\\
%Alex Antonio \textsc{Turriza Suárez} % Your name
%\end{flushleft}
%\end{minipage}
%~
%\begin{minipage}{0.4\textwidth}
%\begin{flushright} \large
%\emph{Asesores:} \\
%Dr. Arturo \textsc{Espinosa Romero} \\
%\ \ \\
%Dr. Anabel \textsc{Martín González}
%\end{flushright}
%\end{minipage}\\[4cm]

%{\large \today}\\[3cm] % Date, change the \today to a set date if you want to be precise

%\includegraphics{Logo}\\[1cm] % Include a department/university logo - this will require the graphicx package

%\vfill % Fill the rest of the page with whitespace

%\end{titlepage}

%----------------------------------------------------------------------------------------
%	TABLE OF CONTENTS
%----------------------------------------------------------------------------------------

%\tableofcontents % Include a table of contents

%\newpage % Begins the essay on a new page instead of on the same page as the table of contents 

%----------------------------------------------------------------------------------------
%	INTRODUCTION
%----------------------------------------------------------------------------------------
\chapter{Sistema de Archivos en Red entre una PC y una BeagleBone Black}\label{Anx:nfs}
\section{Introducción}
Cuando dos máquinas de diferente arquitectura deben trabajar en un sólo proyecto, suele suceder que es mucho más cómodo realizar código y documentación en una, a pesar de que que los archivos estén destinados a ser usados en la otra.

Para ello, se mostrará la forma de configurar un sistema de archivos en red NFS que facilite la tarea de compartir archivos en un directorio.

En este trabajo se mostrará la instalación del sistema en una máquina host en una PC y un cliente en una BeagleBone Black, aprovechando que al conectar mediante USB, se crea una red entre ambas plataformas.

%------------------------------------------------

\section{Definición de NFS}
Sistema de archivos en red (NFS, "\textit{Network File System}" por sus siglas en inglés), es un protocolo que permite acceder mediante una conexión remota a un sistema de archivos [\href{https://debian-handbook.info/browse/es-ES/stable/sect.nfs-file-server.html}{El Manual del Administrador de Debian}\footnotemark, consultado en Octubre 2016].

\footnotetext{\href{https://debian-handbook.info/browse/es-ES/stable/sect.nfs-file-server.html}{https://debian-handbook.info/browse/es-ES/stable/sect.nfs-file-server.html}}

En su funcionamiento, permite que un equipo host comparta determinado directorio con otros equipos clientes, pudiendo determinar qué equipos tienen permisos de lectura, escritura o ambas.

%------------------------------------------------

\section{Descarga e instalación}\label{sec:_install} % Sub-section

\subsection{Instalación en host / PC}
Bajo Ubuntu en sus últimas versiones en el momento de la redacción de éste documento, se abre una terminal con los comandos \textit{Ctrl + Alt + t}.

Lo primero, es actualizar los repositorios con:
\begin{lstlisting}[language=bash]
$ sudo apt-get update
\end{lstlisting}

Una vez actualizados, se procede a la instalación de un paquete mediante el siguiente comando:

\begin{lstlisting}[language=bash]
$ sudo apt-get install nfs-kernel-server 
\end{lstlisting}

Al finalizar la descarga e instalación, se debe modificar un archivo. Copiar en la terminal el siguiente comando y colocar la contraseña:
\begin{lstlisting}[language=bash]
$ sudo nano /etc/default/nfs-kernel-server
\end{lstlisting}

Se debe modificar la línea \textbf{NEED\_SVCGSSD=""} y colocar $"no"$ en el entrecomillado, como muestra la figura \ref{fig:NFSKer}. Cuando se termine de modificar, guardar con la combinación de teclas $Ctrl + O$ y regresar a la terminal con $Ctrl + X$.

\begin{figure}[H] % Example image
\center{\includegraphics[width=0.9\linewidth]{Figures/NFS/NFS1}}
\caption{Archivo /etc/default/nfs-kernel-server ya modificado.}
\label{fig:NFSKer}
\end{figure}

Lo siguiente es abrir el archivo ubicado en /etc/idmapd.conf:
\begin{lstlisting}[language=bash]
$ sudo nano /etc/idmapd.conf 
\end{lstlisting}

Verificar que existan las líneas $Nobody-User = nobody$ y $Nobody-Group = nogroup$ como muestra la figura \ref{fig:NFSKer2}.

\begin{figure}[H] % Example image
\center{\includegraphics[width=0.9\linewidth]{Figures/NFS/NFS2}}
\caption{Archivo /etc/idmapd.conf.}
\label{fig:NFSKer2}
\end{figure}

Cuando se realiza una conexión con la BeagleBone Black mediante un cable USB, se crea una red con las siguientes direcciones: $192.168.7.2$ para la BeagleBone y $192.168.7.1$ para el PC host. Entonces, tomando en cuenta lo anterior, se modifica el archivo /etc/exports de la siguiente manera:

Se abre el archivo con nano, en la terminal:

\begin{lstlisting}[language=bash]
$ sudo nano /etc/exports
\end{lstlisting}

En el archivo que se abre, añadir la siguiente línea (note que dentro del paréntesis, entre los comandos no existen espacios):

\begin{lstlisting}[language=bash]
/home/alexrt07/Escritorio/Alex     192.168.7.2(rw,sync,
no_root_squash,no_subtree_check)
\end{lstlisting}

Donde /home/alexrt07/Escritorio/Alex es el directorio a compartir y 192.168.7.2 es la dirección ip de la BeagleBone. Así, el archivo queda como muestra la figura \ref{fig:NFSKer3}.

\begin{figure}[H] % Example image
\center{\includegraphics[width=0.9\linewidth]{Figures/NFS/NFS3}}
\caption{Archivo /etc/exports.}
\label{fig:NFSKer3}
\end{figure}

Finalmente, resta reiniciar el servidor con el siguiente comando: 

\begin{lstlisting}[language=bash]
$ /etc/init.d/nfs-kernel-server restart
\end{lstlisting}

Se deberá mostrar una confirmación de reinicio exitoso.
%---------------------------------------------------

\subsubsection{Seguridad}\label{subsec:conf}

Para evitar dejar hoyos de seguridad de acceso a los archivos personales, es altamente recomendable modificar los archivos /etc/hosts.deny y /etc/hosts.allow para permitir acceso solamente a los clientes conocidos.

Abrir el archivo /etc/hosts.deny con:

\begin{lstlisting}[language=bash]
$ sudo nano /etc/hosts.deny
\end{lstlisting}

Y añadir la siguiente línea:

\begin{lstlisting}[language=bash]
rpcbind mountd nfsd statd lockd rquotad : ALL
\end{lstlisting}

\begin{figure}[H] % Example image
\center{\includegraphics[width=0.9\linewidth]{Figures/NFS/NFS4}}
\caption{Archivo /etc/hosts.deny}
\label{fig:NFSKer4}
\end{figure}

Como muestra la figura \ref{fig:NFSKer4}. Ahora, abrir el archivo /etc/hosts.allow con el comando:

\begin{lstlisting}[language=bash]
$ sudo nano /etc/hosts.allow
\end{lstlisting}

Y añadir la siguiente línea:

\begin{lstlisting}[language=bash]
  rpcbind mountd nfsd statd lockd 
  rquotad : 192.168.7.2 127.0.0.1 
\end{lstlisting}

Como muestra la figura \ref{fig:NFSKer5}.

\begin{figure}[H] % Example image
\center{\includegraphics[width=0.9\linewidth]{Figures/NFS/NFS5}}
\caption{Archivo /etc/hosts.allow}
\label{fig:NFSKer5}
\end{figure}

Reiniciar el servidor con: 

\begin{lstlisting}[language=bash]
$ service nfs-kernel-server restart
\end{lstlisting}

Ahora, el PC está preparado para compartir vía red el directorio /home/alexrt07/Escritorio/Alex/

\subsection{Instalación en cliente / BeagleBone}

Asumiendo que la BeagleBone Black tiene un Debian con su archivo \textit{/etc/apt/sources.list} correctamente configurado, ejecutamos en la terminal de nuestro host para conectarnos:

\begin{lstlisting}[language=bash]
$ ssh -l root 192.168.7.2
\end{lstlisting}

donde \textit{ssh} es el comando para conectarse por el protocolo secure shell, \textit{-l} es el comando que indica que se hará un login con el usuario \textit{root}, y \textit{192.168.7.2} es la dirección IP de la BeagleBone en la red que se creó a través del cable USB.

Entonces, una vez hecho el loggin, ejecutar:

\begin{lstlisting}[language=bash]
$ apt-get install nfs-common
\end{lstlisting}

Que instalará y preconfigurará los archivos necesarios para una correcta comunicación a través de NFS.

Es recomendable crear un directorio en donde se montarán los archivos que compartirá con la PC:

\begin{lstlisting}[language=bash]
$ mkdir /home/debian/Alex_tesista
\end{lstlisting}

\section{Ejecución}

Se procede a montar el sistema de archivos con el siguiente comando:

\begin{lstlisting}[language=bash]
$ mount -t nfs -o proto=tcp,port=2049 
  192.168.7.1:/home/alexrt07/ARM-Root/home/Alex 
  /home/debian/Alex_tesista/
\end{lstlisting}

En donde \textit{mount} es el comando para montar el directorio, \textit{-t nfs} indica que se trata de un sistema de archivos por red, \textit{-o proto=tcp,port=2049} indica que se utilizará el protocolo de transferencia de archivos a través del puerto 2049 (mirar el manual de nfs en su página 5 con \textbf{\$ man 5 nfs} para más opciones e información),\textit{192.168.7.1:} es la dirección IP de la PC host, \textit{/home/alexrt07/ARM-Root/home/Alex} es el directorio que contiene los archivos a compartir, y \textit{/home/debian/Alex\_tesista/} es el directorio creado en donde se encontrarán los archivos.

Para cerrar esta conexión, utilice

\begin{lstlisting}[language=bash]
$ umount /home/debian/Alex_tesista/
\end{lstlisting}

Tome en cuenta que al finalizar la conexión, no se mantendrán los archivos compartidos por nfs. Desaparecerán y contendrá los archivos originales que esa carpeta contenía antes de montar el sistema de archivos por red.

%\end{document}
%\include{Appendices/AppendixB}
%\include{Appendices/AppendixC}

%----------------------------------------------------------------------------------------
%	BIBLIOGRAPHY
%----------------------------------------------------------------------------------------

%\printbibliography[heading=bibintoc]
\printbibliography

%----------------------------------------------------------------------------------------

\end{document}  
