% Chapter 2

\chapter{Diseño del hardware y software}
\label{Chap:DisHard} % For referencing the chapter elsewhere, use \ref{Chapter2} 

%----------------------------------------------------------------------------------------

\section{Introducción}

Lorem ipsum dolor...

\section{Prototipo}

La interconexión general del sistema quedará como sigue: Es posible observar a la estación base o de tierra en la parte inferior izquierda de la imagen. La estación móvil o Rover se encontrará navegando libremente en un espacio cercano en función a la potencia de los módulos de radiofrecuencia usados para comunicar a ambas plataformas.

\section{Descripción del funcionamiento}

A la estación base se le asignará un conjunto de coordenadas fijas de latitud, longitud y altura sobre el nivel del mar, que es donde ella deberá ser colocada. El GPS realizará las triangulaciones necesarias para obtener su posición estimada, también en coordenadas, a través de los satélites. Así, comparando a las coordenadas previamente fijadas de su posición exacta con las obtenidas de las triangulaciones, la estación base calculará el
error que existe en ese instante en las mediciones y procederá a informar, mediante los módulos de comunicación de radiofrecuencia, a la estación móvil, para que proceda a hacer las correcciones necesarias y obtener una aproximación depurada de su posición y sus movimientos. \\

Se realizó un experimento que consistió en trazar y seguir una determinada ruta para poder observar las diferencias entre un sistema RTK y uno con el GPS triangulando por sí solo. \\

También fueron colocadas marcas en el suelo para poder tener una guía precisa del camino a seguir y repetir en ambos experimentos. \\

Se usaron las coordenadas que Google Maps proporciona, como referencia. De este modo, los resultados irán conforme a los mapas de este servicio. \\

Al final, la ruta a seguir quedaría de este modo:

\section{Descripción del experimento}

Lorem ipsum dolor...

\section{Conclusión}

Lorem ipsum dolor...

En el capítulo siguiente...