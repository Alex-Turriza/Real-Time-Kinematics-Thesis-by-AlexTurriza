% Chapter 1

\chapter{Introducción} % Main chapter title

\label{chap:Intro} % For referencing the chapter elsewhere, use \ref{Intr} 

%----------------------------------------------------------------------------------------

% Define some commands to keep the formatting separated from the content 
\newcommand{\keyword}[1]{\textbf{#1}}
\newcommand{\tabhead}[1]{\textbf{#1}}
\newcommand{\code}[1]{\texttt{#1}}
\newcommand{\file}[1]{\texttt{\bfseries#1}}
\newcommand{\option}[1]{\texttt{\itshape#1}}

%----------------------------------------------------------------------------------------

\section{Planteamiento del Problema}
Los sistemas de medición actuales poseen cierto grado de incertidumbre, que en ocasiones resulta crítico dado el propósito para el que son utilizados, por ejemplo, en sistemas de tiempo real.

\begin{figure}[ht]
\centering
\includegraphics[scale=0.14]{Figures/Pred}
\caption[Mapa Precisión]{Posición aproximada de la ubicación de un celular en un mapa de Google Maps.}
\label{fig:Prec}
\end{figure}

En una planeación de ruta de un sistema de navegación autónomo, una incertidumbre de $n$ metros a la redonda converge en una inexactitud significativa del sistema. El dispositivo puede encontrarse dentro de cualquier punto de la circunferencia de $2n$ metros de diámetro. Revise la figura~\ref{fig:Prec}, donde todo el círculo azul denota la posición aproximada de un dispositivo celular con GPS, y en donde la pequeña elipse roja muestra la posición real en dicho mapa. Además, cuando el sistema no está limitado a sólo seguir una ruta, sino que debe de ejecutar cierta acción cada determinados metros, entonces la incertidumbre se convierte en algo aún más crítico. Es evidente que el sistema no arrojará los resultados esperados debido a la falta de precisión.

\section{Trabajos previos}
En el trabajo \cite{DeLaCruz}, se usa un sistema GPS (Global Positioning System, sistema de posicionamiento global, por sus siglas en inglés) para apoyar en la logística de distribución de farmacéuticos, como control de rutas, tiempos y seguridad, mencionando ante todo la capacidad de poder obtener los datos de localización y tiempo, además de la velocidad de desplazamiento de un vehículo. Sin embargo, de acuerdo al trabajo \cite{Mendoza}, se propone una actualización del ancho de carriles carreteros a una amplitud adecuada máxima de 3.6 m, y dado que un GPS mide su incertidumbre en metros dependiendo del ruido, entonces se está ante un problema que requiere de modelos y cálculos matemáticos extra para reducir dicho error de medición, que podría incluso indicar que el vehículo se encuentra fuera de la cinta asfáltica o circulando en una vía contraria. 

También, la tesis \cite{Ron}, menciona la implementación de un sistema de navegación inercial INS/GPS para un UAV (Unmanned Aerial Vehicle, vehículo aéreo no tripulado, por sus siglas en inglés). Entre los datos a destacar, la posición de este sistema es estimada con un error de 2 metros con un 95\% de confianza, además de otros datos para asistir al vuelo como velocidad y su altitud.

Revisando el desarrollo \cite{Maldonado}, los autores afirman que las coordenadas recibidas de un dispositivo GPS solamente son usadas como aproximación a la posición del vehículo, y nunca como un dato sólido que sirviese al computar datos, dada la exactitud mínima de 10 metros. Sin embargo, mencionan limitantes tales como el peso total de la carga del UAV y el consumo de energía causado por la integración de todos los demás sensores, de donde el GPS aporta entonces una cantidad mínima de información y utilidad en general.

En los trabajos previos se presentan algunos inconvenientes con la precisión del sistema de navegación de los dispositivos. Además, requieren una fuerte cantidad de cálculos matemáticos en filtros que aún así siguen produciendo errores medibles en metros. Por lo tanto, se propone la realización de un sistema de navegación que utilice un dispositivo denominado GPS diferencial, que nominalmente reduce el error a una escala medible en centímetros, ampliando las posibilidades de aprovechamiento de los datos proporcionados por el mismo.

\section{Objetivo}
El objetivo general de este trabajo es el diseño y la implementación de un sistema de navegación capaz de conocer su ubicación en un entorno geográfico con una alta precisión para apoyar en proyectos que requieran de dicha información.
