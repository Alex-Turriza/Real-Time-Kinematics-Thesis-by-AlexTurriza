% Chapter 2

\chapter{Conclusiones} % Main chapter title

\label{Conc} % For referencing the chapter elsewhere, use \ref{Chapter2} 

Durante la realización de este trabajo de tesis, se realizaron diversos experimentos en donde se logró identificar la mayor estabilidad otorgada por Real-Time Kinematics a comparación de un modo donde el GPS funciona sin retroalimentación. Dichas observaciones fueron corroboradas durante la realización del experimento.\\

La BeagleBone Black no presentó ningún inconveniente tanto de consumo de energía como de recursos durante la realización de las rutinas, pudiendo otorgar la solución de posicionamiento tras el procesado de forma óptima.\\

Como consecuencia, se garantiza que la información de posicionamiento del equipo que cuente con un GPS funcionando en modo Real-Time Kinematics será precisa y estable, permitiendo así que sea viable el aplicar control evitando lecturas de baja confiabilidad en la retroalimentación durante el sensado de coordenadas.